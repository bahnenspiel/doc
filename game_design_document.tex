\section{Game Design Document}

\textcolor{red}{\textbf{TODO: Illustration}}

\subsection{Spielobjekte}
Die Spielfigur ist ein Raumschiff, welches sich auf der Bahn bewegen kann. Auf der Bahn befinden sich Hindernisse in Form von Tunnel, Barrikaden, Abgründen oder Bereichen, die nicht befahren werden dürfen. Außerhalb der Bahn befindet sich ebenfalls ein Abgrund.

\subsection{User Interface}
Das User Interface zeigt eine Anzeige für die aktuelle Geschwindigkeit, eine für die Anzahl der verbleibenden Leben, sowie eine weitere Anzeige, die den verbleibenden Kraftstoff anzeigt.

\subsection{Sounds}
Zum jetzigen Zeitpunkt sind keine Sounds geplant. Denkbar sind jedoch Sounds für bestimmte Spielsituationen, wie zum Beispiel das Beenden eines Levels, eine Kollision mit einem Hindernis, Springen oder wenn der Spieler sein letztes Leben verloren hat.

\subsection{Steuerung}
Das Spiel wird mit dem \textit{Smovetec} Fahrradergometer gesteuert. Durch die Trittfrequenz wird die Geschwindigkeit des Raumschiffs beeinflusst. Die Neigung des Ergometers hingegen lässt das Raumschiff nach Links oder Rechts lenken. Um mit dem Raumschiff springen zu können, muss der Spieler eine Taste am Griff des Ergometers betätigen.

\subsection{Levels}
Es gibt eine feste Anzahl von Levels. Mit dem Erreichen eines höheren Levels steigt auch der Schwierigkeitsgrad. Die Level befinden sich in verschiedenen Umgebungen. Diese Umgebungen nehmen Einfluss auf die jeweilige Gravitation, wodurch das Verhalten des Raumschiffs beim Springen beeinflusst wird.