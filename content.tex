\chapter{Game Design Document}
Bei dem Spiel handelt es sich um ein Geschicklichkeitsspiel, bei dem der Spieler ein Raumschiff über einen Parcours im Weltall steuern muss. Die Level bestehen aus einer geraden Bahn auf der der Spieler das Raumschiff nach Links und Rechts bewegen kann indem er sich auf dem Ergometer in die jeweilige Richtung lehnt. Zusätzlich muss der Spieler durch Springen Hindernissen auf der Bahn ausweichen. Durch die Veränderung seiner Trittfrequenz erhöht, bzw. verringert der Spieler die Geschwindigkeit des Raumschiffes. Für die Bewältigung eines Levels gibt es eine vorgegebene Zeitspanne innerhalb der das Ziel erreicht werden muss. Die Zeitspanne wird durch den verbleibenden Kraftstoff des Raumschiffs symbolisiert.\\
Das Level gilt als erfolgreich beendet, wenn der Spieler es schafft mit dem Raumschiff innerhalb der vorgegebenen Zeit das Ziel zu erreichen. Läuft die Zeit ab, kollidiert der Spieler mit einem Hindernis oder kommt von der Bahn ab, so hat der Spieler das Level nicht geschafft, verliert ein Leben und muss es wiederholen. Hat der Spieler alle seine Leben verbraucht kann er seinen Fortschritt in eine Highscoreliste eintragen. Beim nächsten Start des Spiels beginnt er wieder beim ersten Level.

\section{Spielobjekte}
Die Spielfigur ist ein Raumschiff, welches sich auf der Bahn bewegen kann. Auf der Bahn befinden sich Hindernisse in Form von Tunnel, Barrikaden, Abgründen oder Bereichen, die nicht befahren werden dürfen. Außerhalb der Bahn befindet sich ebenfalls ein Abgrund.

\section{User Interface}
Das User Interface zeigt eine Anzeige für die aktuelle Geschwindigkeit, eine für die Anzahl der verbleibenden Leben, sowie eine weitere Anzeige, die den verbleibenden Kraftstoff anzeigt.

\section{Sounds}
Zum jetzigen Zeitpunkt sind keine Sounds geplant. Denkbar sind jedoch Sounds für bestimmte Spielsituationen, wie zum Beispiel das Beenden eines Levels, eine Kollision mit einem Hindernis, Springen oder wenn der Spieler sein letztes Leben verloren hat.

\section{Steuerung}
Das Spiel wird mit dem \textit{Smovetec} Fahrradergometer gesteuert. Durch die Trittfrequenz wird die Geschwindigkeit des Raumschiffs beeinflusst. Die Neigung des Ergometers hingegen lässt das Raumschiff nach Links oder Rechts lenken. Um mit dem Raumschiff springen zu können, muss der Spieler eine Taste am Griff des Ergometers betätigen.

\section{Levels}
Es gibt eine feste Anzahl von Levels. Mit dem Erreichen eines höheren Levels steigt auch der Schwierigkeitsgrad. Die Level befinden sich in verschiedenen Umgebungen. Diese Umgebungen nehmen Einfluss auf die jeweilige Gravitation, wodurch das Verhalten des Raumschiffs beim Springen beeinflusst wird.