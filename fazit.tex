\chapter{Zusammenfassung und Ausblick}
Das Ziel des Projektes war es, ein Spiel zur Förderung des Trainings des Oberkörpers auf einem Fahrradergometer zu konzipieren und umzusetzen. Hierfür wurde zunächst eine umfassende Recherche möglicher Spielideen und der Hardware durchgeführt. Anhand dieser Recherche konnte ein Spielkonzept inklusive Spielbeschreibung erstellt werden. Die entsprechende Umsetzung wurde in weiteren Schritten geplant und dokumentiert, bevor die eigentliche Implementierung begonnen wurde. \\
Bei der Umsetzung konnten sämtliche geplanten Features realisiert und getestet werden, dennoch bietet das, in diesem Praktikum abgeschlossene Projekt, einige Möglichkeiten zur Erweiterung der Funktionalität. Es wäre zum Beispiel denkbar dem Spieler zusätzliche Anreize innerhalb der Level zu bieten. So könnte man bestimmte Objekte auf den Bahnen platzieren durch welche der Spieler Vorteile innerhalb des aktuellen, beziehungsweise der nächsten Levels hat oder wodurch der Spieler neue Fähigkeiten und/oder Objekte freischalten kann. \\
Eine weiter Möglichkeit das Spiel stetig zu erweitern bietet der realisierte Leveleditor. Durch die einfache Integration in das Projekt können zukünftige Entwickler mit sehr geringem Aufwand neue Level Objekte erstellen und dadurch den Umfang des Spieles schnell erheblich vergrößern. Ebenfalls mit Hilfe des Leveleditors können schnell neue Levels erstellt werden, die anschließend durch die verschiedenen Texturen und Hintergründe nach bestimmten Themen sortiert werden können. Es wäre zum Beispiel denkbar ein Thema mit stark erhöhter Gravitation zu erstellen, welches sich über eine gewisse Anzahl von Levels erstreckt. Dadurch können, ohne dass neue Spielobjekte oder -mechaniken erstellt werden müssten neue Anreize zu spielen gesetzt werden.\\
Um den kompetetiven Gedanken der Spieler zu fördern wäre eine Highscoreliste dienlich. Statt dem einzigen Ziel das Spiel erfolgreich zu beenden, könnte man zusätzlich Faktoren wie zum Beispiel benötigte Zeit, benötigte Versuche, verbleibende Energy, oder andere in einer Liste speichern. Hierfür sind jedoch noch weitere Anpassungen notwendig, um die genannten Faktoren erfassen und Speichern zu können.\\
Die wohl weitreichendsten Änderungen könnten jedoch mit der Personalisation des Spieles an den Spieler erreicht werden. Zwar bietet das Projekt zum jetzigen Stand bereits die Möglichkeit die gemessene Geschwindigkeit und die Sensitivität des Ergometers an den jeweiligen Spieler anzupassen, allerdings werden diese Einstellungen noch nicht gespeichert. Denkbar hierfür wäre eine Art Benutzerverwaltung, in der verschiedene Spieler ein Profil hinterlegen können welches ihre Einstellungen speichert. Im Zuge dessen könnte ebenfalls der Spielfortschritt gespeichert werden. \\
Zusammenfassend kann man festhalten, dass im Rahmen dieses Praktikums ein Bewegungsspiel konzipiert und implementiert wurde, dass alle relevanten Punkte der Aufgabenstellung einhält und darüber hinaus zusätzlich noch sehr viel Raum für eventuelle weitere Entwicklungsschritte bietet. Durch die realisierten Adaptionsmechanismen eignet sich das Spiel für eine große Bandbreite von Zielgruppen und besitzt dabei das Potenzial spielerisch zur Gesundheitsförderung beitragen. Dieser Aspekt sollte in naher Zukunft durch eine Benutzerstudie evaluiert werden.
