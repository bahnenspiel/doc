\section{Zielsetzung und Anforderungen}
Das Ziel des Projekts ist es, ein Spiel zu entwickeln, das die Möglichkeiten des Smovetec-Fahrradergometers im Bereich Videospiele demonstriert. Insbesondere ist liegt also der Fokus des Projekts darin, eine Spielidee zu finden, welche die Steuerungsmechanismen Trittfrequenz und Neigung mit einem spaßigen Konzept verbindet.\\
Die wichtigste Anforderung an das Projekt ist also eine sinnvolle Nutzung des Fahrradergometers als Eingabegerät. Da das eingesetzte Fahrradergometer nicht unbedingt immer zur Verfügung steht, ist eine Ersatzsteuerung nur mit Maus und Tastatur sinnvoll.\\
Eine weitere wichtige Anforderung ist außerdem, dass das entstandene Spiel Spaß machen soll. Da das Spiel jedoch vor allem zu Demonstrationszwecken dienen soll, ist es insbesondere wichtig, dass das Spiel nicht kompliziert zu erlernen ist und auch für alle Altersstufen geeignet ist, zumindest im Rahmen der vom Fahrradergometer vorgegebenen Grenzen. Für Kinder unter 6 Jahren wird das Spiel somit vermutlich nicht geeignet sein, da für das Ergometer eine gewisse Körpergröße erforderlich ist.\\
Aus dem Einsatzzweck heraus ergibt sich auch, dass eine längere Bindung von Spielern, wie beispielsweise durch eine Story, ein Highscore-System oder durch einen Mehrspielermodus nicht notwendig ist.\\
Die Zielgruppe ist somit jeder, insbesondere auch ohne Videospiel-Erfahrung, im Alter ab 6 Jahren.
\section{Features}
\begin{itemize}
\item Steuerung durch Trittfrequenz und Neigung
\item Ersatzsteuerung mit Maus und Tastatur
\item Ansprechende optische Darstellung
\item Langsam steigender Schwierigkeitsgrad
\item Untermalung mit Musik und Geräuschen
\item Verschiedene Designs der Spielumgebung
\end{itemize}


\section{Priorisierung}
Oberste Priorität hat die Steuerung mittels Fahrradergometer. Direkt danach kommen eine ansprechende Präsentation und eine gewisse Zahl an Levels, welche einen gemächlich steigenden Schwierigkeitsgrad abbilden.\\
Die Ersatzsteuerung, Musik und verschiedene Designs gehören eher zu den Features mit niedriger Priorität.