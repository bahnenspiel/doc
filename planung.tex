\chapter{Planung}
\section{Projektplanung}
\begingroup
\small
\begin{tabularx}{\textwidth}{|X|c|c|c|} \hline
 &  & \multicolumn{2}{c|}{\textbf{Wer?}} \\ \hline
\textbf{Teilobjekte und Arbeitspakete} & \textbf{Summe (Personentage)} & \textbf{Alex} & \textbf{Simon} \\ \hline
\textbf{1. Koordination und Organisation} & \textbf{7} & \textbf{3} & \textbf{4} \\ \hline
1.1. Einrichten der Entwicklungsumgebung & 2 & 1 & 1 \\ \hline
1.2. Einrichten des Dokumentationstemplates & 2 & 1 & 1 \\ \hline
1.3 Einrichten der Repositories & 1 & 0 & 1 \\ \hline
1.4. Zeitplanung & 2 & 1 & 1 \\ \hline
\textbf{2. Konzeption und Findung der Spielidee} & \textbf{7} & \textbf{5} & \textbf{2} \\ \hline
2.1. Findung der Spielidee & 3 & 2 & 1 \\ \hline
2.2. Game Design Document & 2 & 2 & 0 \\ \hline
2.3. Recherche zur Hardware & 2 & 1 & 1 \\ \hline
\textbf{3. Recherche} & \textbf{8} & \textbf{5} & \textbf{3} \\ \hline
3.1. Ähnliche Spielideen recherchieren & 2 & 1 & 1 \\ \hline
3.2. Spiele mit ähnlichen Eingabegeräten recherchieren & 4 & 3 & 1 \\ \hline
3.3. Geeignete Werkzeuge und Technologien & 2 & 1 & 1 \\ \hline
\textbf{4. Entwicklung des Spielgrundgerüsts} & \textbf{7} & \textbf{4} & \textbf{3} \\ \hline
4.1. Rudimentäres Leveldesign & 2 & 1 & 1 \\ \hline
4.2. Einfache Steuerung per Tastatur & 2 & 1 & 1 \\ \hline
4.3. Einfache graphische Darstellung & 2 & 1 & 1 \\ \hline
4.4. Einfaches graphische Benutzerschnittstelle & 1 & 1 & 0 \\ \hline
\textbf{5. Leveldesign} & \textbf{21} & \textbf{9} & \textbf{12} \\ \hline
5.1. Entwicklung eines Levelgesamtkonzepts & 7 & 3 & 4 \\ \hline
5.2. Gestaltung von Grundelementen für die Levels & 6 & 3 & 3 \\ \hline
5.3. Erstellung einer größeren Anzahl Levels & 8 & 3 & 5 \\ \hline
\textbf{6. Steuerung} & \textbf{15} & \textbf{7} & \textbf{8} \\ \hline
6.1. Anbindung der Trittfrequenz & 3 & 1 & 2 \\ \hline
6.2. Anbindung der Fahrradneigung & 8 & 4 & 4 \\ \hline
6.3. Anbindung eines Aktionsbuttons & 4 & 2 & 2 \\ \hline
\textbf{7. Graphische Darstellung} & \textbf{17} & \textbf{9} & \textbf{8} \\ \hline
7.1. Gestaltung von Models und Texturen & 10 & 5 & 5 \\ \hline
7.2. Gestaltung der Spielumgebung & 4 & 2 & 2 \\ \hline
7.3. Gestaltung von Effekten & 3 & 2 & 1 \\ \hline
\textbf{8. GUI} & \textbf{10} & \textbf{4} & \textbf{6} \\ \hline
8.1. Entwicklung eines GUI Konzepts & 2 & 1 & 1 \\ \hline
8.2. Gestaltung der GUI-Elemente & 4 & 1 & 3 \\ \hline
8.3. Implementierung der GUI im Spiel & 4 & 2 & 2 \\ \hline
\textbf{9. Benutzerdokumentation} & \textbf{7} & \textbf{3} & \textbf{4} \\ \hline
\textbf{10. Entwicklerdokumentation} & \textbf{7} & \textbf{4} & \textbf{3} \\ \hline
 &  &  &  \\ \hline
Gesamt & 92 & 46 & 46 \\ \hline
\end{tabularx}

\newenvironment{ganttap}[5]{
\ganttgroup{#1}{2015-#3-#2}{2015-#5-#4} \ganttnewline
}{
\ganttnewline[thick]}


\section{Zeitplanung}
\resizebox{\textwidth}{!}{

\begin{ganttchart}[
x unit=1.2mm,
y unit title=7mm,
y unit chart=7mm,
group label node/.append style={align=left, text width=15em},
bar label node/.append style={align=left, text width=15em},
milestone label node/.append style={align=left, text width=15em},
vgrid={*4{draw=none},*1{dotted},*2{draw=none}},
time slot format=isodate
]{2015-04-22}{2015-09-20}
\gantttitlecalendar{month=shortname} \\
\gantttitle{1}{5}
\gantttitlelist{2,...,22}{7}\\

\ganttmilestone[inline=true]{Milestone 1}{2015-06-21}

\ganttnewline[thick]

\begin{ganttap}{1. Koordination}{22}{04}{29}{04}
\ganttbar{1.1. Zeitplanung}{2015-04-22}{2015-04-26} \ganttnewline
\ganttbar{1.2. Einrichten Template}{2015-04-22}{2015-04-26} \ganttnewline
\ganttbar{1.3. Einrichten Repo}{2015-04-27}{2015-04-29} \ganttnewline
\ganttbar{1.4. Einrichten IDE}{2015-04-27}{2015-04-29}
\end{ganttap}

\begin{ganttap}{2. Konzeption}{22}{04}{29}{04}
\ganttbar{2.1. Spielidee}{2015-04-22}{2015-04-29} \ganttnewline
\ganttbar{2.2. Design Document}{2015-04-22}{2015-04-26} \ganttnewline
\ganttbar{2.3. Recherche Hardware}{2015-04-27}{2015-04-29}
\end{ganttap}

\begin{ganttap}{3. Recherche}{30}{04}{17}{05}
\ganttbar{3.1. Spielideen}{2015-04-30}{2015-05-03} \ganttnewline
\ganttbar{3.2. Eingabegeräte}{2015-05-04}{2015-05-10} \ganttnewline
\ganttbar{3.3. Werkzeuge}{2015-05-11}{2015-05-17}
\end{ganttap}

\begin{ganttap}{4. Spielgrundgerüst}{30}{04}{12}{05}
\ganttbar{4.1. Leveldesign}{2015-04-30}{2015-05-03} \ganttnewline
\ganttbar{4.2. Tastatursteuerung}{2015-04-30}{2015-05-12} \ganttnewline
\ganttbar{4.3. Graphik}{2015-04-30}{2015-05-12} \ganttnewline
\ganttbar{4.4. GUI}{2015-04-30}{2015-05-12}
\end{ganttap}

\begin{ganttap}{5. Leveldesign}{18}{05}{15}{09}
\ganttbar{5.1. Gesamtkonzept}{2015-05-18}{2015-05-24} \ganttnewline
\ganttbar{5.2. Grundelemente}{2015-06-01}{2015-06-14}
\ganttbar{}{2015-07-06}{2015-07-12} \ganttnewline
\ganttbar{5.3. Levelerstellung}{2015-06-15}{2015-06-28}
\ganttbar{}{2015-07-20}{2015-07-26}
\ganttbar{}{2015-08-10}{2015-08-23} \ganttnewline
\ganttbar{5.4. Testen \& Balancing}{2015-06-29}{2015-07-05}
\ganttbar{}{2015-07-27}{2015-08-02}
\ganttbar{}{2015-08-24}{2015-09-13}
\end{ganttap}

\begin{ganttap}{6. Steuerung}{13}{05}{23}{06}
\ganttbar{6.1. Trittfrequenz}{2015-05-13}{2015-05-24} \ganttnewline
\ganttbar{6.2. Neigung}{2015-05-25}{2015-06-14} \ganttnewline
\ganttbar{6.3. Aktionsbutton}{2015-06-15}{2015-06-23}
\end{ganttap}

\begin{ganttap}{7. Gestaltung}{24}{06}{15}{09}
\ganttbar{7.1. Models \& Texturen}{2015-06-24}{2015-07-05}
\ganttbar{}{2015-07-13}{2015-07-26}
\ganttbar{}{2015-08-03}{2015-08-16} \ganttnewline
\ganttbar{7.2. Umgebung}{2015-08-24}{2015-09-06} \ganttnewline
\ganttbar{7.3. Effekte}{2015-09-07}{2015-09-13}
\end{ganttap}

\begin{ganttap}{8. GUI}{29}{06}{19}{07}
\ganttbar{8.1. Konzept}{2015-06-29}{2015-07-05} \ganttnewline
\ganttbar{8.2. Gestaltung}{2015-07-06}{2015-07-19} \ganttnewline
\ganttbar{8.3. Implementierung}{2015-07-06}{2015-07-19}
\end{ganttap}

\ganttgroup{9. Benutzerdokumentation}{2015-08-17}{2015-09-15}
\ganttnewline[thick]
\ganttgroup{10. Entwicklerdokumentation}{2015-04-30}{2015-09-15}


\end{ganttchart}

}
\section{Meilensteine}
\paragraph{Projektdefinition -- 29.4.}
\noindent
Die Projektdefinition enthält die \textit{tabellarische Planung (vgl. \ref{ap1})} des Projektes, das \textit{Gantt-Diagramm (vgl. \ref{ap1})} und das \textit{Game Design Document (vgl. \ref{ap2})}. Ziel ist es, eine erste Übersicht über die Spielidee, den geschätzten Aufwand und die zeitliche Planung des Projektes zu gewinnen.
\paragraph{Mock-up -- 13.5.}
\noindent
Das Mock-up soll dem Projektteam und dem Betreuer eine bereits spielbare Veranschaulichung der Spielidee bieten. Daher enthält dieser Meilenstein das wesentliche \textit{Spielgrundgerüst (vgl. \ref{ap4})}.
\paragraph{Alpha-Version -- 23.6.}
\noindent
Die Alpha-Version soll bereits erahnen lassen, wie das Spiel zu Projektende aussehen soll. Der Umfang ist aber noch sehr eingeschränkt. Zu diesem Meilenstein gehören ein \textit{einfaches Levelpaket (vgl. \ref{ap5})}, die \textit{Steuerung (vgl. \ref{ap6})}, das \textit{Spielermodell und ein einfaches Texturenpaket (vgl. \ref{ap7})}.
\paragraph{Beta-Version -- 19.7.}
\noindent
In der Beta-Version soll das Spiel dann schon alle Elemente enthalten, die es auch in der fertigen Version hat, wenn auch in geringerem Umfang. Daher kommt in dieser Version noch eine \textit{graphische Benutzeroberfläche (vgl. \ref{ap8})} hinzu.
\paragraph{Pre-finale Ausarbeitung -- 2.9.}
\noindent
Die pre-finale Ausarbeitung enthält bereits alle Inhalte der finalen Ausarbeitung, muss aber noch gegebenenfalls überarbeitet werden. Außer den Inhalten der Projektdefinition enthält diese eine \textit{Recherche (vgl. \ref{ap3})}, sowie die \textit{Benutzer- und Entwicklerdokumentation (vgl. \ref{ap9}, \ref{ap10})}.
\paragraph{Finale Version -- 15.9.}
\noindent
Zur finalen Version werden alle noch für das fertige Spiel fehlenden Arbeiten erledigt. Dies beinhaltet ein \textit{vollständiges Levelpaket (vgl. \ref{ap5})} und \textit{vollständige Texturen, Umgebungen und Effekte (vgl. \ref{ap7})}.
\section{Arbeitspakte}
\newcolumntype{Y}{>{\centering\arraybackslash}X}

\newcommand{\aphead}[5]{\begin{tabularx}{\textwidth}{lYr}
\textbf{Beginn:} #1 \hspace{0.5cm} \textbf{Ende:} #2 & \textbf{Lead:} #3 \hspace{0.5cm} \textbf{Beteiligte:} #4 & \textbf{Aufwand:} #5
\end{tabularx}
\hrule}

\subsection{Koordination und Organisation}
\aphead{22.4.}{29.4.}{alle}{alle}{7 Personentage}

\paragraph{Zielstellung}\noindent
Ziel des Arbeitspaketes \textit{Koordination und Organisation} ist es, die Grundlagen für einen erfolgreichen Start des Projektes zu legen. Nach Abschluss des Arbeitspaketes Koordination und Organisation sollen die Gruppenmitglieder in der Lage sein, die Arbeit am eigentlichen Produkt aufnehmen zu können.

\paragraph{Beschreibung}\noindent
Das Arbeitspaket \textit{Koordination und Organisation} gliedert sich in vier Teilobjekte. Im ersten Teilobjekt wird die Zeitplanung des Projektablaufs vorgenommen. Hierfür wird neben einer tabellarischen Gliederung der Arbeitspakete ein Gantt-Diagramm des zeitlichen Ablaufs erstellt.\\
Anschließend werden Templates für das Projektmanagementdokument eingerichtet und angepasst. Dafür, und für den Quellcode des Projekts, werden im dritten Teilobjekt Versionsverwaltungsrepositories mit Git angelegt und eingerichtet. Abschließend beginnen die Projektbeteiligten damit, die nötigen Entwicklungsumgebungen und Tools einzurichten.

\paragraph{Rolle der Beteiligten}\noindent
Da das Arbeitspaket \textit{Koordination und Organisation} für den gesamten Projektablauf entscheidend ist, sind alle Projektmitglieder hier gleichermaßen beteiligt. Dabei ist jedes Mitglied für die Einrichtung seiner Werkzeuge selbst verantwortlich.

\paragraph{Deliverables}\noindent
\begin{itemize}
\item Tabellarischer Projektplan, Dokument, bis 29.4.
\item Gantt-Diagramm, Dokument, bis 29.4.
\end{itemize}

\paragraph{Meilensteine}\noindent
\begin{itemize}
\item Der \textit{tabellarische Projektplan} und das \textit{Gantt-Diagramm} sind Teil des Meilensteins \textit{Projektdefinition}.
\end{itemize}

\subsection{Konzeption und Findung der Spielidee}
\aphead{22.4.}{29.4.}{Alex}{alle}{7 Personentage}

\paragraph{Zielstellung}\noindent
Ziel des Arbeitspaketes \textit{Konzeption und Findung der Spielidee} ist es, ein Konzept für ein Bewegungsspiel mit einem Fahrradergometer bedient werden soll, zu entwickeln. Außerdem soll ein \textit{Game Design Document}, welches das Konzept genauer erläutert, verfasst werden.

\paragraph{Beschreibung}\noindent
Das Arbeitspaket \textit{Konzeption und Findung der Spielidee} untergliedert sich in drei Teilobjekte. Zunächst erarbeiten die Projektmitglieder eine auf die Aufgabenstellung passende Spielidee. Das gefundene Konzept wird im zweiten Teilobjekt detailliert ausgearbeitet und als \textit{Game Design Document} ausformuliert. Da die bereitgestellte Hardware in Form des Eingabegeräts Fahrradergometer für die Projektbeteiligten unbekannt ist, wird außerdem eine umfassende Recherche zu den technischen Möglichkeiten und Einschränkungen durchgeführt. 

\paragraph{Rolle der Beteiligten}\noindent
Alle Mitglieder des Projektes erarbeiten gemeinsam das Konzept für die Spielidee, sowie die Recherche zur Hardware. Das \textit{Game Design Document} wird überwiegend vom Hauptverantwortlichen des Arbeitspaketes verfasst.

\paragraph{Deliverables}\noindent
\begin{itemize}
\item \textit{Game Design Document}, Dokument, bis 29.4.
\end{itemize}

\paragraph{Meilensteine}\noindent
\begin{itemize}
\item Das \textit{Game Design Document} ist Teil des Meilensteins \textit{Projektdefinition}.
\end{itemize}
\subsection{Recherche}
\aphead{22.4.}{1.9.}{Alex}{alle}{8 Personentage}

\paragraph{Zielstellung}\noindent
Ziel des Arbeitspaketes \textit{Recherche} ist es, eine Übersicht über einige, bereits existierende, Spiele mit vergleichbarer Steuerung beziehungsweise ähnlicher Spielidee zu gewinnen. Außerdem sollen für die Umsetzung des Projektes geeignete Technologien und Werkzeuge gesucht und bewertet werden.

\paragraph{Beschreibung}\noindent
Das Arbeitspaket \textit{Recherche} gliedert sich in drei Teilobjekte, wobei sich die Pakete zur Recherche von ähnlichen Spielideen und Spielen mit ähnlichen Eingabegeräten in ihrer Ausgestaltung überschneiden. Der Fokus der Recherche liegt sowohl auf kommerziellen Produkten, als auch auf wissenschaftlichen Projekten.\\
Nach Abschluss des Arbeitspaketes \textit{Konzeption und Findung der Spielidee} wird in diesem Arbeitspaket eine Recherche der für die Umsetzung der Spielidee passenden Werkzeuge, Spielengines und Technologien durchgeführt. Darauf aufbauend wird eine Auswahl geeigneter Werkzeuge getroffen.

\paragraph{Rolle der Beteiligten}\noindent
Der Leiter des Arbeitspaketes übernimmt die Recherche bestehender Spiele und Systeme. Der zweite Projektteilnehmer führt parallel die Recherche über Technologien und Werkzeuge durch.

\paragraph{Deliverables}\noindent
\begin{itemize}
\item State-of-the-art-Recherche zu Spielen und Systemen, Dokument, bis 2.9.
\end{itemize}

\paragraph{Meilensteine}\noindent
\begin{itemize}
\item Der Bericht zur Recherche ist Teil des Meilensteins \textit{Pre-finale Version}.
\end{itemize}

\subsection{Entwicklung des Spielgrundgerüsts}
\aphead{30.4.}{12.5.}{alle}{alle}{7 Personentage}

\paragraph{Zielstellung}\noindent
Ziel des Arbeitspaketes \textit{Entwicklung des Spielgrundgerüsts} ist es, nach der Findung des Konzeptes eine prototypische Implementierung des Konzeptes zu erstellen und zu testen. Zweck des Prototypen ist es, als Machbarkeitsstudie des zuvor entworfenen Konzeptes zu dienen.

\paragraph{Beschreibung}\noindent
Das Arbeitspaket \textit{Entwicklung des Spielgrundgerüsts} gliedert sich in die vier Teilaspekte Levelbau, Steuerung, Grafik und graphische Nutzerschnittstelle.\\
Dabei soll das Level nur minimal die Idee des Spiels widerspiegeln und die grundlegende Spielmechanik ermöglichen. Die Steuerung soll zunächst nur über die Tastatur erfolgen, um den Entwicklungsaufwand gering zu halten. Die eigentliche Steuerung mit Fahrradergometer wird in einem späteren Arbeitspaket realisiert. Im Zuge des Levelbaus werden einfache 3D-Modelle als Levelobjekte erstellt und mit einfachen Texturen versehen. Abschließend wird eine rudimentäre graphische Nutzerschnittstelle programmiert, die hauptsächlich zur Veranschaulichung der Funktionalität der späteren Nutzerschnittstelle dient.

\paragraph{Rolle der Beteiligten}\noindent
Damit alle Beteiligten einen ähnlichen Wissensstand bezüglich der verwendeten Technologien erreichen, erfolgt die Entwicklung in diesem Arbeitspaket ausschließlich gemeinsam nach dem Prinzip des \textit{Pair Programming}\cite[S. 42ff]{xp}.

\paragraph{Deliverables}\noindent
\begin{itemize}
\item Spielgrundgerüst, Software, bis 13.5.
\end{itemize}

\paragraph{Meilensteine}\noindent
\begin{itemize}
\item Das Spielgrundgerüst ist Teil des Meilensteins \textit{Mock-Up}.
\end{itemize}

\subsection{Leveldesign}
\aphead{18.5.}{15.9.}{Simon}{alle}{27 Personentage}

\paragraph{Zielstellung}\noindent
Ziel des Arbeitspaketes \textit{Leveldesign} ist es, eine möglichst hohe Anzahl interessanter Levels zu erstellen, welche zunehmend schwerer und fordernder werden. Abschließend sollen die Levels getestet und balanciert werden.

\paragraph{Beschreibung}\noindent
Das Arbeitspaket \textit{Leveldesign} gliedert sich in vier Teilpakete. Im ersten Schritt wird ein grundlegendes Levelgesamtkonzept erarbeitet. Basierend auf diesem Konzept werden anschließend die einzelnen Levels und im Zuge dessen die Levelgrundobjekte, wie zum Beispiel Bahnen, Rampen und Tunnel erstellt. Parallel dazu wird jedes Level nach der Fertigstellung ausgiebig evaluiert und gegebenenfalls angepasst. Nach Abschluss der Erstellung aller Levels werden diese in Bezug auf Probleme des Schwierigkeitsgrades analysiert und überarbeitet. Hierdurch soll eine angenehme Lernkurve für den Spieler gewährleistet werden.

\paragraph{Rolle der Beteiligten}\noindent
Alle Projektbeteiligten sind gleichermaßen an der Erstellung, Evaluierung und Balancing der Levels beteiligt. Die Erstellung des Gesamtkonzepts wird federführend vom Lead des Arbeitspakets durchgeführt. Die Erstellung der Spielobjekte und Levels erfolgt zu Beginn gemeinsam, um ein einheitliches Gesamtbild zu gewährleisten. Später wird die Levelerstellung getrennt durchgeführt, wodurch die Effizienz gesteigert werden soll.

\paragraph{Deliverables}\noindent
\begin{itemize}
\item Einfaches Levelpaket, Software, bis 23.6.
\item Vollständiges Levelpaket, Software, bis 15.9.
\end{itemize}

\paragraph{Meilensteine}\noindent
\begin{itemize}
\item Das einfache Levelpaket ist Teil des Meilensteins \textit{Alphaversion}.
\item Das vollständige Levelpaket ist Teil des Meilensteins \textit{Pre-finale Version}.
\end{itemize}
\subsection{Steuerung}
\label{ap6}
\aphead{13.5.}{23.6.}{Simon}{alle}{15 Personentage}

\paragraph{Zielstellung}\noindent
Ziel des Arbeitspaketes \textit{Steuerung} ist es, alle für das Spiel notwendigen Steuerungsmechanismen zu implementieren und zu testen. Dies sind die Trittfrequenz und Neigung des Fahrradergometers und eine am Lenker angebrachte Aktionstaste.

\paragraph{Beschreibung}\noindent
Das Arbeitspaket \textit{Steuerung} gliedert sich entsprechend der Eingabemöglichkeiten in drei Teilobjekte, die zeitlich parallel bearbeitet werden können. Durch die Trittfrequenz, die der Spieler auf dem Fahrradergometer leistet, wird die Geschwindigkeit des Raumschiffes im Spiel gesteuert. Die Neigung des Fahrradergometers wird gemessen, um Richtungswechsel des Raumschiffes im Level zu vollziehen. Zusätzlich wird eine Aktionstaste am Lenker des Fahrradergometers dazu verwendet um springen zu können. Hierdurch kann der Spieler Hindernissen ausweichen oder die Fahrbahn wechseln.\\
Alle drei Steuerungsmechanismen werden ausgiebig getestet, um einen angenehmen und flüssigen Spielablauf zu garantieren.

\paragraph{Rolle der Beteiligten}\noindent
Da mit einigen Schwierigkeiten bei der Anbindung der externen Steuerungshardware gerechnet wird, erfolgt die Umsetzung dieses Arbeitspaketes gemeinsam und nach dem Prinzip des \textit{Pair Programming}\cite[S. 42ff]{xp}.

\paragraph{Deliverables}\noindent
\begin{itemize}
\item Implementierung der Steuerung, Software, bis 23.6.
\end{itemize}

\paragraph{Meilensteine}\noindent
\begin{itemize}
\item Die Steuerung ist Teil des Meilensteins \textit{Alphaversion}.
\end{itemize}
\subsection{Audiovisuelle Gestaltung}
\aphead{13.5.}{15.9.}{Alex}{alle}{17 Personentage}

\paragraph{Zielstellung}\noindent
Ziel des Arbeitspaketes \textit{Audiovisuelle Gestaltung} ist es, das Spiel graphisch zu gestalten. Dies umfasst das Spielermodell, Texturen, die Spielumgebung und Effekte. Nach Abschluss des Paketes sollte das Spiel einen stimmigen und ansprechenden graphischen Gesamteindruck vermitteln.

\paragraph{Beschreibung}\noindent
Das Arbeitspaket \textit{Audiovisuelle Gestaltung} gliedert sich drei Teilpakete: \textit{Gestaltung von Models und Texturen}, \textit{Gestaltung der Spielumgebung} und \textit{Gestaltung von Effekten}. Das erste Teilpaket umfasst vor allem die graphische Gesamtgestaltung, das zweite Teilpaket die Erstellung einer dazu passenden Spielumgebung. Abschließend wird das Spielerlebnis durch ansprechende audiovisuelle Effekte, wie zum Beispiel Explosionen, angereichert.

\paragraph{Rolle der Beteiligten}\noindent

\paragraph{Deliverables}\noindent
\begin{itemize}
\item 
\end{itemize}

\paragraph{Meilensteine}\noindent
\begin{itemize}
\item 
\end{itemize}
