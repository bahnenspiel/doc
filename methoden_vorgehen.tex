\chapter{Methoden und Vorgehen}
Bei der Durchführung des Projekts hat sich das Team an einem agilen Entwicklungsmodell orientiert. Das bedeutet, dass man nicht zu Beginn eine ausführliche Planung aller Features bis ins Detail durchgeführt hat, sondern, dass vielmehr jede Idee und jeden Ansatz direkt als Mock-up im Spiel ausprobiert wurde, um schnelles und exaktes Feedback über die Machbarkeit und Sinnhaftigkeit des Features zu erlangen.\\
Dieses Vorgehen hat den Effekt, dass manche Features begonnen, aber dann nachdem man festgestellt hat, dass sie so nicht machbar oder nicht zielführend sind, wieder verworfen wurden. Auf der anderen Seite ermöglicht dieses Vorgehen aber auch ein sehr schnelles Reagieren auf neue Begebenheiten. Insbesondere wird so sichergestellt, dass keine überflüssigen Features erst aufwändig geplant werden und sich dann in der Realität als obsolet herausstellen. Aber insbesondere ermöglicht diese Methode ein sehr schnelles Lernen aus Feedback um so ein für den Nutzer optimales Spielerlebnis zu schaffen.\\
\paragraph{Controlling}\noindent
Um den Fortschritt im Projekt und auch mögliche Rückschläge immer möglichst gut im Blick zu behalten, bestreiten wir einen möglichst großen Teil der Entwicklung immer mit dem Team gemeinsam. Wenn dies zeitweise nicht möglich ist, so halten wir uns per E-Mail detailliert auf dem Laufenden. Dennoch ist ein gegenseitiges Update beim nächsten persönlichen Treffen unumgänglich.\\
So oft möglich erfolgt die Entwicklung in den Räumlichkeiten des Lehrstuhls, sodass bei Schwierigkeiten, aber auch für regelmäßiges Feedback der Betreuer unkompliziert hinzugezogen werden kann.
\paragraph{„lessons-learned“}\noindent
Ein genaues Durchführen und Einhalten eines vorab erstellten Plans ist in so einem Softwareprojekt nicht möglich. Gerade wenn man wenig Erfahrung mit den eingesetzten Technologien hat, ist ein Abschätzen der Aufwände nicht einfach und äußerst ungenau. Daher hat es sich am praktikabelsten erwiesen, von Tag zu Tag neu zu priorisieren und zu entscheiden. Trotzdem konnte man das große Ganze immer gut im Blick behalten und man wusste immer in etwa, wo man im Projekt stehen.
