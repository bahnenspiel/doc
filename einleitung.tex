\chapter{Einleitung}
Seit geraumer Zeit findet man auf dem Spielemarkt immer mehr Spiele die durch den eigenen Körper gesteuert werden können. Bekannte Beispiele dafür umfassen die Spielekonsole Wii der Firma Nintendo, sowie der Bewegungssensor Kinect von Microsoft für die Konsole Xbox. Viele dieser Spiele bedürfen entweder den Einsatz des Oberkörpers, beziehungsweise der Arme, wenige andere werden durch Fahrradergometer gesteuert, wie zum Beispiel ErgoActive der TU Darmstadt [vgl. Kapitel \ref{StateOfTheArt}]. Bei beiden Arten wird jedoch stets nur ein Teil des Körpers zur Steuerung genutzt. Eine Kombination aus beiden Steuerungen bietet \textit{Smovetec}. Smovetec ist eine Plattform auf die Fahrradergometer montiert werden können, um dadurch zusätzlich zur Trittfrequenz auch die Neigung beim Fahrradfahren zu simulieren. Allerdings gibt es noch keine Spiele, die diese Möglichkeit als zusätzliche Steuerung nutzen. \\
In diesem Projekt soll ein solches Spiel realisiert werden. Beim „Bahnenspiel“ steuert der Spieler ein Raumschiff in einer fiktionalen Weltraum-Umgebung über enge Bahnen, die mit Hindernissen gespickt sind und immer im Kampf gegen die Uhr, dargestellt über den verbleibenden „Treibstoff“ des Raumschiffs. Das Besondere des Spiels ist die ausgefallene Steuerung, die mit dem Smovetec System ermöglicht wird. Statt mit Maus und Tastatur am Schreibtisch, gibt man die Lenkimpulse und Geschwindigkeit für das Raumschiff mit dem Fahrradergometer vor.\\
Ziel des Spiels ist es aber nicht, möglichst schnell durch die Strecken zu fahren, sondern vielmehr die immer schwieriger und komplizierter werdenden Levels überhaupt in der vorgegebenen Zeit zu meistern. Dazu ist weniger die körperliche Fitness gefordert, sondern vor allem die Geschicklichkeit, das Gleichgewicht zu halten und gleichzeitig gerade genug in die richtige Richtung zu steuern um den Hindernissen auszuweichen. Zusätzlich zu der Steuerung mittels Neigung und Trittfrequenz kommt später im Spiel auch noch das drücken einer Taste am Lenker hinzu, womit ein Sprung ausgelöst wird, der an manchen Stellen notwendig ist, um einem großen Hindernis zu entgehen. \\
Dieses Dokument beschreibt zunächst das Projekt mit allen Anforderungen und Zielgruppen, anschließend wird das Konzept und die Architektur des Projektes, sowie die daraus resultierende Planung in Bezug auf Zeitplanung, Meilensteine und Arbeitspakete erläutert.
In einem weiteren Kapitel werden ähnliche Spielideen und Eingabemethoden betrachtet. Ebenfalls behandelt das Dokument die eigentliche Umsetzung der Projektidee. Bevor ein Fazit des Projektes gezogen und auf mögliche Erweiterungen des Spieles eingegangen wird, findet man eine Beschreibung zur Steuerung des Spiels, sowie nötige Hinweise um eine eigene Version des Spiels mit eigenen Levels erstellen zu können.
