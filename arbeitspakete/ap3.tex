\subsection{Recherche}
\aphead{30.4.}{17.5.}{Alex}{alle}{8 Personentage}

\paragraph{Zielstellung}\noindent
Ziel des Arbeitspaketes \textit{Recherche} ist es, eine Übersicht über einige, bereits existierende, Spiele mit vergleichbarer Steuerung beziehungsweise ähnlicher Spielidee zu gewinnen. Außerdem sollen für die Umsetzung des Projektes geeignete Technologien und Werkzeuge gesucht und bewertet werden.

\paragraph{Beschreibung}\noindent
Das Arbeitspaket \textit{Recherche} gliedert sich in drei Teilobjekte, wobei sich die Pakete zur Recherche von ähnlichen Spielideen und Spielen mit ähnlichen Eingabegeräten in ihrer Ausgestaltung überschneiden. Der Fokus der Recherche liegt sowohl auf kommerziellen Produkten, als auch auf wissenschaftlichen Projekten.\\
Nach Abschluss des Arbeitspaketes \textit{Konzeption und Findung der Spielidee} wird in diesem Arbeitspaket eine Recherche der für die Umsetzung der Spielidee passenden Werkzeuge, Spielengines und Technologien durchgeführt. Darauf aufbauend wird eine Auswahl geeigneter Werkzeuge getroffen.

\paragraph{Rolle der Beteiligten}\noindent
Der Leiter des Arbeitspaketes übernimmt die Recherche bestehender Spiele und Systeme. Der zweite Projektteilnehmer führt parallel die Recherche über Technologien und Werkzeuge durch.

\paragraph{Deliverables}\noindent
\begin{itemize}
\item State-of-the-art-Recherche zu Spielen und Systemen, Dokument, bis 2.9.
\end{itemize}

\paragraph{Meilensteine}\noindent
\begin{itemize}
\item Der Bericht zur Recherche ist Teil des Meilensteins \textit{Pre-finale Version}.
\end{itemize}
