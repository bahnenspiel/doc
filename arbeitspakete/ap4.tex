\subsection{Entwicklung des Spielgrundgerüsts}
\label{ap4}
\aphead{30.4.}{12.5.}{alle}{alle}{7 Personentage}

\paragraph{Zielstellung}\noindent
Ziel des Arbeitspaketes \textit{Entwicklung des Spielgrundgerüsts} ist es, nach der Findung des Konzeptes eine prototypische Implementierung des Konzeptes zu erstellen und zu testen. Zweck des Prototypen ist es, als Machbarkeitsstudie des zuvor entworfenen Konzeptes zu dienen.

\paragraph{Beschreibung}\noindent
Das Arbeitspaket \textit{Entwicklung des Spielgrundgerüsts} gliedert sich in die vier Teilaspekte Levelbau, Steuerung, Grafik und graphische Nutzerschnittstelle.\\
Dabei soll das Level nur minimal die Idee des Spiels widerspiegeln und die grundlegende Spielmechanik ermöglichen. Die Steuerung soll zunächst nur über die Tastatur erfolgen, um den Entwicklungsaufwand gering zu halten. Die eigentliche Steuerung mit Fahrradergometer wird in einem späteren Arbeitspaket realisiert. Im Zuge des Levelbaus werden einfache 3D-Modelle als Levelobjekte erstellt und mit einfachen Texturen versehen. Abschließend wird eine rudimentäre graphische Nutzerschnittstelle programmiert, die hauptsächlich zur Veranschaulichung der Funktionalität der späteren Nutzerschnittstelle dient.

\paragraph{Rolle der Beteiligten}\noindent
Damit alle Beteiligten einen ähnlichen Wissensstand bezüglich der verwendeten Technologien erreichen, erfolgt die Entwicklung in diesem Arbeitspaket ausschließlich gemeinsam nach dem Prinzip des \textit{Pair Programming}\cite[S. 42ff]{xp}.

\paragraph{Deliverables}\noindent
\begin{itemize}
\item Spielgrundgerüst, Software, bis 13.5.
\end{itemize}

\paragraph{Meilensteine}\noindent
\begin{itemize}
\item Das Spielgrundgerüst ist Teil des Meilensteins \textit{Mock-Up}.
\end{itemize}
