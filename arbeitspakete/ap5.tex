\subsection{Leveldesign}
\aphead{13.5.}{15.9.}{Simon}{alle}{27 Personentage}

\paragraph{Zielstellung}\noindent
Ziel des Arbeitspaketes \textit{Leveldesign} ist es, eine möglichst hohe Anzahl interessanter Levels zu erstellen, welche zunehmend schwerer und fordernder werden. Abschließend sollen die Levels getestet und balanciert werden.

\paragraph{Beschreibung}\noindent
Das Arbeitspaket \textit{Leveldesign} gliedert sich in vier Teilpakete. Im ersten Schritt wird ein grundlegendes Levelgesamtkonzept erarbeitet. Basierend auf diesem Konzept werden anschließend die einzelnen Levels und im Zuge dessen die Levelgrundobjekte, wie zum Beispiel Bahnen, Rampen und Tunnel erstellt. Parallel dazu wird jedes Level nach der Fertigstellung ausgiebig evaluiert und gegebenenfalls angepasst. Nach Abschluss der Erstellung aller Levels werden diese in Bezug auf Probleme des Schwierigkeitsgrades analysiert und überarbeitet. Hierdurch soll eine angenehme Lernkurve für den Spieler gewährleistet werden.

\paragraph{Rolle der Beteiligten}\noindent
Alle Projektbeteiligten sind gleichermaßen an der Erstellung, Evaluierung und Balancing der Levels beteiligt. Die Erstellung des Gesamtkonzepts wird federführend vom Lead des Arbeitspakets durchgeführt. Die Erstellung der Spielobjekte und Levels erfolgt zu Beginn gemeinsam, um ein einheitliches Gesamtbild zu gewährleisten. Später wird die Levelerstellung getrennt durchgeführt, wodurch die Effizienz gesteigert werden soll.

\paragraph{Deliverables}\noindent
\begin{itemize}
\item Einfaches Levelpaket, Software, bis 23.6.
\item Vollständiges Levelpaket, Software, bis 15.9.
\end{itemize}

\paragraph{Meilensteine}\noindent
\begin{itemize}
\item Das einfache Levelpaket ist Teil des Meilensteins \textit{Alphaversion}.
\item Das vollständige Levelpaket ist Teil des Meilensteins \textit{Pre-finale Version}.
\end{itemize}