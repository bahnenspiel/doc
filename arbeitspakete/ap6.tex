\subsection{Steuerung}
\label{ap6}
\aphead{13.5.}{23.6.}{Simon}{alle}{15 Personentage}

\paragraph{Zielstellung}\noindent
Ziel des Arbeitspaketes \textit{Steuerung} ist es, alle für das Spiel notwendigen Steuerungsmechanismen zu implementieren und zu testen. Dies sind die Trittfrequenz und Neigung des Fahrradergometers und eine am Lenker angebrachte Aktionstaste.

\paragraph{Beschreibung}\noindent
Das Arbeitspaket \textit{Steuerung} gliedert sich entsprechend der Eingabemöglichkeiten in drei Teilobjekte, die zeitlich parallel bearbeitet werden können. Durch die Trittfrequenz, die der Spieler auf dem Fahrradergometer leistet, wird die Geschwindigkeit des Raumschiffes im Spiel gesteuert. Die Neigung des Fahrradergometers wird gemessen, um Richtungswechsel des Raumschiffes im Level zu vollziehen. Zusätzlich wird eine Aktionstaste am Lenker des Fahrradergometers dazu verwendet um springen zu können. Hierdurch kann der Spieler Hindernissen ausweichen oder die Fahrbahn wechseln.\\
Alle drei Steuerungsmechanismen werden ausgiebig getestet, um einen angenehmen und flüssigen Spielablauf zu garantieren.

\paragraph{Rolle der Beteiligten}\noindent
Da mit einigen Schwierigkeiten bei der Anbindung der externen Steuerungshardware gerechnet wird, erfolgt die Umsetzung dieses Arbeitspaketes gemeinsam und nach dem Prinzip des \textit{Pair Programming}\cite[S. 42ff]{xp}.

\paragraph{Deliverables}\noindent
\begin{itemize}
\item Implementierung der Steuerung, Software, bis 23.6.
\end{itemize}

\paragraph{Meilensteine}\noindent
\begin{itemize}
\item Die Steuerung ist Teil des Meilensteins \textit{Alphaversion}.
\end{itemize}