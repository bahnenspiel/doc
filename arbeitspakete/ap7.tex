\subsection{Audiovisuelle Gestaltung}
\aphead{24.6.}{15.9.}{Alex}{alle}{17 Personentage}

\paragraph{Zielstellung}\noindent
Ziel des Arbeitspaketes \textit{Audiovisuelle Gestaltung} ist es, das Spiel graphisch zu gestalten. Dies umfasst das Spielermodell, Texturen, die Spielumgebung und Effekte. Nach Abschluss des Paketes sollte das Spiel einen stimmigen und ansprechenden graphischen Gesamteindruck vermitteln.

\paragraph{Beschreibung}\noindent
Das Arbeitspaket \textit{Audiovisuelle Gestaltung} gliedert sich drei Teilpakete: \textit{Gestaltung von Models und Texturen}, \textit{Gestaltung der Spielumgebung} und \textit{Gestaltung von Effekten}. Das erste Teilpaket umfasst vor allem die graphische Gesamtgestaltung, das zweite Teilpaket die Erstellung einer dazu passenden Spielumgebung. Abschließend wird das Spielerlebnis durch ansprechende audiovisuelle Effekte, wie zum Beispiel Explosionen, angereichert.

\paragraph{Rolle der Beteiligten}\noindent
Alle Projektbeteiligten werden bei der Gestaltung der Spielelemente einbezogen, da diese sehr wichtig für den Gesamteindruck des Spiels ist. Die Modellierung von 3D-Modellen wird dabei insbesondere vom Hauptverantwortlichen des Arbeitspaketes durchgeführt.

\paragraph{Deliverables}\noindent
\begin{itemize}
\item Spielermodell, Software, bis 23.6.
\item Einfaches Texturenpaket, Software, bis 23.6.
\item Vollständige Texturensammlung, Software, bis 15.9.
\item Spielumgebung, Software, bis 15.9.
\item Effekte, Software, bis 15.9.
\end{itemize}

\paragraph{Meilensteine}\noindent
\begin{itemize}
\item Das Spielermodell und das einfache Texturenpaket sind Bestandteile des Meilensteins \textit{Alphaversion}.
\item Das vollständige Texturenpaket, die Spielumgebung und die Effekte sind Teil des Meilensteins \textit{Pre-finale Version}
\end{itemize}