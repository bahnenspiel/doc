\subsection{Audiovisuelle Gestaltung}
\aphead{13.5.}{15.9.}{Alex}{alle}{17 Personentage}

\paragraph{Zielstellung}\noindent
Ziel des Arbeitspaketes \textit{Audiovisuelle Gestaltung} ist es, das Spiel graphisch zu gestalten. Dies umfasst das Spielermodell, Texturen, die Spielumgebung und Effekte. Nach Abschluss des Paketes sollte das Spiel einen stimmigen und ansprechenden graphischen Gesamteindruck vermitteln.

\paragraph{Beschreibung}\noindent
Das Arbeitspaket \textit{Audiovisuelle Gestaltung} gliedert sich drei Teilpakete: \textit{Gestaltung von Models und Texturen}, \textit{Gestaltung der Spielumgebung} und \textit{Gestaltung von Effekten}. Das erste Teilpaket umfasst vor allem die graphische Gesamtgestaltung, das zweite Teilpaket die Erstellung einer dazu passenden Spielumgebung. Abschließend wird das Spielerlebnis durch ansprechende audiovisuelle Effekte, wie zum Beispiel Explosionen, angereichert.

\paragraph{Rolle der Beteiligten}\noindent

\paragraph{Deliverables}\noindent
\begin{itemize}
\item 
\end{itemize}

\paragraph{Meilensteine}\noindent
\begin{itemize}
\item 
\end{itemize}