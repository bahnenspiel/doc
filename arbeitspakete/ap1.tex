\subsection{Koordination und Organisation}
\aphead{22.4.}{29.4.}{alle}{alle}{7 Personentage}

\paragraph{Zielstellung}\noindent
Ziel des Arbeitspaketes \textit{Koordination und Organisation} ist es, die Grundlagen für einen erfolgreichen Start des Projektes zu legen. Nach Abschluss des Arbeitspaketes Koordination und Organisation sollen die Gruppenmitglieder in der Lage sein, die Arbeit am eigentlichen Produkt aufnehmen zu können.

\paragraph{Beschreibung}\noindent
Das Arbeitspaket \textit{Koordination und Organisation} gliedert sich in vier Teilobjekte. Im ersten Teilobjekt wird die Zeitplanung des Projektablaufs vorgenommen. Hierfür wird neben einer tabellarischen Gliederung der Arbeitspakete ein Gantt-Diagramm des zeitlichen Ablaufs erstellt.\\
Anschließend werden Templates für das Projektmanagementdokument eingerichtet und angepasst. Dafür, und für den Quellcode des Projekts, werden im dritten Teilobjekt Versionsverwaltungsrepositories mit Git angelegt und eingerichtet. Abschließend beginnen die Projektbeteiligten damit, die nötigen Entwicklungsumgebungen und Tools einzurichten.

\paragraph{Rolle der Beteiligten}\noindent
Da das Arbeitspaket \textit{Koordination und Organisation} für den gesamten Projektablauf entscheidend ist, sind alle Projektmitglieder hier gleichermaßen beteiligt. Dabei ist jedes Mitglied für die Einrichtung seiner Werkzeuge selbst verantwortlich.

\paragraph{Deliverables}\noindent
\begin{itemize}
\item Tabellarischer Projektplan, Dokument, bis 29.4.
\item Gantt-Diagramm, Dokument, bis 29.4.
\end{itemize}

\paragraph{Meilensteine}\noindent
\begin{itemize}
\item Der \textit{tabellarische Projektplan} und das \textit{Gantt-Diagramm} sind Teil des Meilensteins \textit{Projektdefinition}.
\end{itemize}
