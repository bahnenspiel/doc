\subsection{GUI}
\aphead{29.6.}{19.7.}{Simon}{alle}{10 Personentage}

\paragraph{Zielstellung}\noindent
Ziel des Arbeitspaketes \textit{GUI} ist es, eine Benutzeroberfläche zu dem eigentlichen Spiel zu erstellen, welche dem Spieler Informationen zum aktuellen Spielzustand darstellt. Die GUI sollte gut ablesbar sein und sich thematisch an der Gesamtgestaltung des Spiels orientieren.

\paragraph{Beschreibung}\noindent
Das Arbeitspaket \textit{GUI} gliedert sich drei Teilpakete: Zuerst wird ein Konzept entworfen, wie die Benutzeroberfläche aussehen soll. Im zweiten Teilpaket werden die einzelnen Elemente gestaltet und die erforderlichen Grafiken erstellt. Parallel dazu wird auch an der Implementierung und Verknüpfung mit dem Spiel gearbeitet. Dabei wird auch getestet, ob sich die Oberflächenelemente gut ablesen lassen und notfalls noch korrigierend eingegriffen.

\paragraph{Rolle der Beteiligten}\noindent


\paragraph{Deliverables}\noindent
\begin{itemize}
\item 
\end{itemize}

\paragraph{Meilensteine}\noindent
\begin{itemize}
\item 
\end{itemize}