\subsection{Entwicklerdokumentation}
\aphead{30.4.}{15.9.}{Simon}{alle}{7 Personentage}

\paragraph{Zielstellung}\noindent
Ziel des Arbeitspaketes \textit{Entwicklerdokumentation} ist es, eine umfassende Dokumentation des Projektes für nachfolgende Entwicklungsschritte bereit zu stellen. Vor allem Entwickler, denen das Projekt unbekannt ist, sollen sich schnell im Konzept und der Architektur zurechtfinden. Darüber hinaus soll die Entwicklerdokumentation einen Überblick über das Projektmanagement während der gesamten Laufzeit bieten.

\paragraph{Beschreibung}\noindent
Das Arbeitspaket \textit{Entwicklerdokumentation} ist nicht weiter unterteilt, stattdessen wird nach Fertigstellung jedes Arbeitspaketes die dazugehörige Entwicklerdokumentation erstellt. Abschließend werden die einzelnen Kapitel in eine ansprechende Gesamtdokumentation zusammengefasst und Korrektur gelesen.

\paragraph{Rolle der Beteiligten}\noindent
Die Verantwortlichen der jeweiligen Arbeitspakete erstellen die entsprechenden Kapitel der Dokumentation. Sollte ein Arbeitspaket keinen Hauptverantwortlichen haben, so wird diese Aufgabe von allen Projektbeteiligten gleichermaßen bearbeitet.\\
Der Verantwortliche des Arbeitspaketes \textit{Entwicklerdokumentation} fasst die jeweiligen Abschnitte zusammen in eine Gesamtdokumentation und beseitigt eventuelle Rechtschreibfehler.

\paragraph{Deliverables}\noindent
\begin{itemize}
\item Entwicklerdokumentation, Dokument, bis 15.9.
\end{itemize}

\paragraph{Meilensteine}\noindent
\begin{itemize}
\item Die Entwicklerdokumentation ist Bestandteil des Meilensteins \textit{Finale Version}.
\end{itemize}