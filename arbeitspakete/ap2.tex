\subsection{Konzeption und Findung der Spielidee}
\aphead{22.4.}{29.4.}{Alex}{alle}{7 Personentage}

\paragraph{Zielstellung}\noindent
Ziel des Arbeitspaketes \textit{Konzeption und Findung der Spielidee} ist es, ein Konzept für ein Bewegungsspiel mit einem Fahrradergometer bedient werden soll, zu entwickeln. Außerdem soll ein \textit{Game Design Document}, welches das Konzept genauer erläutert, verfasst werden.

\paragraph{Beschreibung}\noindent
Das Arbeitspaket \textit{Konzeption und Findung der Spielidee} untergliedert sich in drei Teilobjekte. Zunächst erarbeiten die Projektmitglieder eine auf die Aufgabenstellung passende Spielidee. Das gefundene Konzept wird im zweiten Teilobjekt detailliert ausgearbeitet und als \textit{Game Design Document} ausformuliert. Da die bereitgestellte Hardware in Form des Eingabegeräts Fahrradergometer für die Projektbeteiligten unbekannt ist, wird außerdem eine umfassende Recherche zu den technischen Möglichkeiten und Einschränkungen durchgeführt. 

\paragraph{Rolle der Beteiligten}\noindent
Alle Mitglieder des Projektes erarbeiten gemeinsam das Konzept für die Spielidee, sowie die Recherche zur Hardware. Das \textit{Game Design Document} wird überwiegend vom Hauptverantwortlichen des Arbeitspaketes verfasst.

\paragraph{Deliverables}\noindent
\begin{itemize}
\item \textit{Game Design Document}, Dokument, bis 29.4.
\end{itemize}

\paragraph{Meilensteine}\noindent
\begin{itemize}
\item Das \textit{Game Design Document} ist Teil des Meilensteins \textit{Projektdefinition}.
\end{itemize}