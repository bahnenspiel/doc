\subsection{Benutzerdokumentation}
\aphead{23.6.}{15.9.}{Alex}{alle}{7 Personentage}

\paragraph{Zielstellung}\noindent
Ziel des Arbeitspaketes \textit{Benutzerdokumentation} ist es, eine verständliche Dokumentation des Spieles für den Endanwender bereit zu stellen. Sie soll vollständig, aber dennoch einen für den Benutzer annehmbaren Umfang haben, sowie leicht verständlich sein.

\paragraph{Beschreibung}\noindent
Das Arbeitspaket \textit{Benutzerdokumentation} ist nicht weiter unterteilt, stattdessen wird nach Fertigstellung jedes, für den Benutzer relevanten, Arbeitspaketes die dazugehörige Benutzerdokumentation erstellt. Abschließend werden die einzelnen Kapitel in eine ansprechende Gesamtdokumentation zusammengefasst und Korrektur gelesen.

\paragraph{Rolle der Beteiligten}\noindent
Die Verantwortlichen der jeweiligen Arbeitspakete erstellen die entsprechenden Kapitel der Dokumentation. Sollte ein Arbeitspaket keinen Hauptverantwortlichen haben, so wird diese Aufgabe von allen Projektbeteiligten gleichermaßen bearbeitet.\\
Der Verantwortliche des Arbeitspaketes \textit{Benutzerdokumentation} fasst die jeweiligen Abschnitte zusammen in eine Gesamtdokumentation und beseitigt eventuelle Rechtschreibfehler.

\paragraph{Deliverables}\noindent
\begin{itemize}
\item Benutzerdokumentation, Dokument, bis 15.9.
\end{itemize}

\paragraph{Meilensteine}\noindent
\begin{itemize}
\item Die Benutzerdokumentation ist Bestandteil des Meilensteins \textit{Finale Version}.
\end{itemize}